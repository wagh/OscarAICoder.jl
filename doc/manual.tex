\documentclass[11pt,a4paper]{article}
\usepackage{amsmath}
\usepackage{amsfonts}
\usepackage{amssymb}
\usepackage{graphicx}
\usepackage{makeidx}
\usepackage{imakeidx}
\usepackage{listings}
\usepackage{color}
\usepackage{url}
\usepackage{fancyhdr}
\usepackage{geometry}
\usepackage{xcolor}
\usepackage{enumitem}
\usepackage{array}
\usepackage{hyperref}

% Index configuration
\makeindex% [columns=1, title=Index, intoc]
% \indexsetup{level=\section*,toclevel=section}
% \makeatletter
% \patchcmd{\theindex}{\section*{\indexname}}{\section*{\indexname}\indexmarkup}{}{}
% \makeatother

% Custom index entries with proper formatting
\newcommand{\indexentry}[2]{\index{#1@\texttt{#1} #2}}
\newcommand{\indexfunc}[1]{\indexentry{#1}{function}}
\newcommand{\indexparam}[1]{\indexentry{#1}{parameter}}
\newcommand{\indexopt}[1]{\indexentry{#1}{option}}
\newcommand{\indexconst}[1]{\indexentry{#1}{constant}}
\newcommand{\indexmod}[1]{\indexentry{#1}{module}}

% Index formatting
\renewcommand{\indexname}{Index}
\renewcommand{\indexspace}{\vspace{12pt}}
% \renewcommand{\indexfont}{\bfseries}

% Command for code in text
\newcommand{\code}[1]{\texttt{#1}}

% Command for important notes
\newcommand{\important}[1]{\textbf{#1}}

% Command for configuration options
\newcommand{\configopt}[1]{\texttt{#1}}

% Command for function signatures
\newcommand{\func}[1]{\texttt{#1}}

% Command for module names
\newcommand{\modname}[1]{\texttt{#1}}

% Page layout
\geometry{left=2cm,right=2cm,top=2.5cm,bottom=2.5cm}
\setlength{\headheight}{14pt}
\setlength{\parindent}{0pt}
\setlength{\parskip}{6pt}

% Headers and footers
\pagestyle{fancy}
\fancyhf{}
\fancyhead[L]{\leftmark}
\fancyhead[R]{\thepage}
\fancyfoot[C]{\thepage}

% Define Julia language for listings
\lstdefinelanguage{Julia}{
    morekeywords={abstract,break,case,catch,continue,do,else,elseif,end,export,for,function,immutable,import,if,in,macro,module,otherwise,quote,return,switch,type,typealias,using,while},
    morecomment=[l]{\#},
    morecomment=[n]{\#=}{=\#},
    morestring=[s]{"}{"},
    morestring=[m]{'}{'}
}

% Custom colors
\definecolor{codegreen}{rgb}{0,0.6,0}
\definecolor{codegray}{rgb}{0.5,0.5,0.5}
\definecolor{codepurple}{rgb}{0.58,0,0.82}
\definecolor{backcolour}{rgb}{0.95,0.95,0.92}
\definecolor{lightgray}{rgb}{0.95,0.95,0.95}

% Listings style
\lstdefinestyle{mystyle}{
    backgroundcolor=\color{backcolour},
    commentstyle=\color{codegreen},
    keywordstyle=\color{magenta},
    numberstyle=\tiny\color{codegray},
    stringstyle=\color{codepurple},
    basicstyle=\footnotesize\ttfamily,
    breakatwhitespace=false,
    breaklines=true,
    captionpos=b,
    keepspaces=true,
    numbers=left,
    numbersep=5pt,
    showspaces=false,
    showstringspaces=false,
    showtabs=false,
    tabsize=2,
    language=Julia,
    frame=single,
    rulecolor=\color{gray!30}
}

\lstset{style=mystyle}

% Custom commands
% Define custom commands if they don't already exist
\providecommand{\code}[1]{\texttt{\color{blue!70!black}#1}}
\providecommand{\important}[1]{\textbf{\color{red!70!black}#1}}

\title{OscarAICoder.jl Documentation}
\author{OscarAICoder Development Team}
\date{\today}

\begin{document}

\maketitle
\thispagestyle{empty}

\begin{abstract}
  This document provides comprehensive documentation for the
  \modname{OscarAICoder.jl} package, a powerful tool for generating Oscar
  \cite{oscar_project} code from natural language mathematical statements. The
  package supports multiple backends, context management, and various
  configuration options to customize the code generation process.
\end{abstract}

\section{Introduction}
\label{sec:introduction}

\modname{OscarAICoder.jl} is a Julia package that provides an interface to interact with large language models (LLMs) for generating Oscar code from natural language mathematical statements. The package is designed to be flexible, supporting multiple backends and configuration options to suit various use cases.

\section{Installation}
\label{sec:installation}

\subsection{Basic Requirements}
- Julia 1.6 or later
- Oscar.jl package
- HTTP.jl and JSON.jl packages

\subsection{Installation Methods}

\subsubsection{1. Project Environment Installation (Recommended)}

\paragraph{Steps:}
1. Clone the repository:
\begin{lstlisting}[language=bash]
git clone https://github.com/wagh/OscarAICoder.jl.git
\end{lstlisting}

2. Navigate to the project directory:
\begin{lstlisting}[language=bash]
cd OscarAICoder.jl
\end{lstlisting}

3. Install the package and its dependencies (only needed once):
\begin{lstlisting}[language=Julia]
import Pkg
Pkg.activate(".")  # Activate the project environment
Pkg.instantiate()  # Install all dependencies (only needed once)
\end{lstlisting}

4. In any new Julia session:
\begin{lstlisting}[language=Julia]
import Pkg
Pkg.activate(".")  # Activate the project environment
using OscarAICoder
\end{lstlisting}

\paragraph{Why it's recommended:}
Project environment installation is the recommended approach because:
\begin{itemize}
    \item Prevents dependency conflicts with other packages
    \item Ensures consistent behavior across different projects
    \item Makes it easier to manage different versions of the package
    \item Follows Julia's best practices for package development
    \item Makes it easier for others to use your package with the exact same dependencies
\end{itemize}

\subsubsection{2. Global Installation (Not Recommended)}

\paragraph{Steps:}
If you want to use \code{using OscarAICoder} directly without activating any environment:

1. Clone the repository:
\begin{lstlisting}[language=bash]
git clone https://github.com/wagh/OscarAICoder.jl.git
\end{lstlisting}

2. Install globally (only run once):
\begin{lstlisting}[language=Julia]
import Pkg
Pkg.develop(path="/path/to/OscarAICoder")
\end{lstlisting}

3. In any new Julia session:
\begin{lstlisting}[language=Julia]
using OscarAICoder
\end{lstlisting}

\paragraph{Why it's not recommended:}
Global installation is not recommended because:
\begin{itemize}
    \item Can cause dependency conflicts with other packages
    \item Makes it harder to manage different versions of the package
    \item Can interfere with package development
    \item Makes it harder to ensure consistent behavior across different systems
    \item Makes maintenance more difficult
\end{itemize}

\subsection{Setting Up LLM Backends}

\subsubsection{Local LLM Server (Recommended)}
1. Install Ollama:
\begin{lstlisting}[language=bash]
wget https://raw.githubusercontent.com/wagh/OscarAICoder.jl/main/setup_ollama.sh
chmod +x setup_ollama.sh
./setup_ollama.sh
\end{lstlisting}

2. After installation:
\begin{lstlisting}[language=Julia]
using OscarAICoder
oscar_code = process_statement("Factor the polynomial x^2 - 5x + 6 over the integers.")
\end{lstlisting}

\section{Basic Usage}
\label{sec:basic_usage}

\subsection{Processing Statements}

The main function to process natural language statements is \func{process\_statement}:

\begin{lstlisting}
using OscarAICoder

# Process a mathematical statement
oscar_code = process_statement("Find the roots of x^2 - 4x + 4")

# Execute the code
result = execute_statement(oscar_code)

# View history
entries = History.get_entries()
\end{lstlisting}

\subsection{Executing Generated Code}
\label{sec:executing_code}

The \modname{OscarAICoder.jl} package provides functions to execute the Oscar code generated by \func{process\_statement}.

\subsubsection{\func{execute\_statement\_with\_format}}
\indexfunc{execute\_statement\_with\_format}

This is the primary function for executing Oscar code and obtaining a formatted string representation of the result. It is available as \func{OscarAICoder.execute\_statement\_with\_format} (exported from the \modname{Core} module).

\paragraph{Signature:}
\begin{lstlisting}[language=Julia, basicstyle=\footnotesize\ttfamily\color{black}, keywordstyle=\color{blue}, commentstyle=\color{green!50!black}, stringstyle=\color{red}]
function execute_statement_with_format(
    oscar_code::String;
    output_format::Symbol = :string
)
\end{lstlisting}

\paragraph{Purpose:}
Executes a string of Oscar code and returns the result formatted according to the \code{output\_format} parameter.

\paragraph{Parameters:}
\begin{itemize}
    \item \code{oscar\_code::String}: (Required) A string containing the Oscar computer algebra system code to execute.
    \item \code{output\_format::Symbol}: (Keyword, defaults to \code{:string}) Specifies the desired format for the output string.
    \begin{itemize}
        \item \code{:string}: (Default) Converts the result to a plain string using Julia's \code{string()} function.
        \indexopt{output\_format=:string}
        \item \code{:latex}: Converts the result into a LaTeX string using \code{Oscar.latex()}. Ideal for mathematical typesetting.
        \indexopt{output\_format=:latex}
        \item \code{:html}: Converts the result into an HTML string using \code{Oscar.html()}. Useful for web display.
        \indexopt{output\_format=:html}
        \item Any other symbol: Defaults to the \code{:string} format.
    \end{itemize}
\end{itemize}

\paragraph{Return Value:}
A \code{String} containing the formatted result of the executed Oscar code.

\paragraph{Error Handling:}
If an error occurs during execution or formatting, the function throws an \code{ErrorException} with details about the failure. It is recommended to wrap calls in a \code{try...catch} block for robust error handling.

\paragraph{Examples:}
Let's assume the following Oscar code:
\begin{lstlisting}[language=Julia, basicstyle=\footnotesize\ttfamily\color{black}]
oscar_cmd = "R, x = PolynomialRing(QQ, \"x\"); p = (x+1)^2"
// This defines p as x^2 + 2x + 1
\end{lstlisting}

\begin{enumerate}
    \item \textbf{Using \code{output\_format=:string} (Default):}
    \begin{lstlisting}[language=Julia]
str_result = execute_statement_with_format(oscar_cmd)
// Expected str_result (illustrative): "x^2 + 2*x + 1"
    \end{lstlisting}

    \item \textbf{Using \code{output\_format=:latex}:}
    \begin{lstlisting}[language=Julia]
latex_result = execute_statement_with_format(oscar_cmd; output_format=:latex)
// Expected latex_result (illustrative): "x^{2} + 2x + 1"
    \end{lstlisting}

    \item \textbf{Using \code{output\_format=:html}:}
    \begin{lstlisting}[language=Julia]
html_result = execute_statement_with_format(oscar_cmd; output_format=:html)
// Expected html_result (illustrative): "x<sup>2</sup> + 2x + 1"
    \end{lstlisting}
\end{enumerate}

\subsubsection{\func{execute\_statement} (from \modname{OscarAICoder.execute})}
\indexfunc{execute\_statement}

The \modname{OscarAICoder.execute} module (whose functions are also available under \modname{OscarAICoder}) provides two versions of \func{execute\_statement}.

\paragraph{1. Returning Raw Object:}
\begin{lstlisting}[language=Julia, basicstyle=\footnotesize\ttfamily\color{black}, keywordstyle=\color{blue}, commentstyle=\color{green!50!black}, stringstyle=\color{red}]
function execute_statement(oscar_code::String)
\end{lstlisting}
\begin{itemize}
    \item \textbf{Purpose:} Executes the Oscar code and returns the \important{raw Julia/Oscar object} resulting from the computation, not a formatted string. This is useful if you need to perform further operations on the result.
    \item \textbf{Return Value:} The actual object (e.g., a polynomial, matrix, number).
    \item \textbf{Error Handling:} Throws an \code{ErrorException} on failure.
    \item \textbf{Example:}
    \begin{lstlisting}[language=Julia]
// result_obj will be an Oscar polynomial object
result_obj = execute_statement("R, x = PolynomialRing(QQ, \"x\"); (x+1)^2")
// You can then format it manually:
// latex_str = Oscar.latex(result_obj)
    \end{lstlisting}
\end{itemize}

\paragraph{2. Returning Formatted String (Alternative):}
\begin{lstlisting}[language=Julia, basicstyle=\footnotesize\ttfamily\color{black}, keywordstyle=\color{blue}, commentstyle=\color{green!50!black}, stringstyle=\color{red}]
function execute_statement(
    oscar_code::String;
    output_format::Symbol = :string
)
\end{lstlisting}
\begin{itemize}
    \item \textbf{Purpose:} This function also executes Oscar code and returns a formatted string, serving as an alternative to \func{execute\_statement\_with\_format} from the \modname{Core} module.
    \item \textbf{Parameters, Return Value, and Error Handling:} For details on the \code{output\_format} options (\code{:string}, \code{:latex}, \code{:html}), specific return values, error handling, and usage examples, please refer to the comprehensive documentation for \func{execute\_statement\_with\_format} in the preceding section.
    \item \textbf{Primary Recommendation:} For consistency and clarity, using \func{OscarAICoder.execute\_statement\_with\_format} (from the \modname{Core} module) is generally recommended when a formatted string output is desired.
\end{itemize}

It is important to choose the correct execution function based on whether you need the raw computational object or a string representation for display or logging.


\subsection{Using Different Backends}

The package supports multiple backends for code generation:

\begin{itemize}
    \item \texttt{LOCAL}: Uses local Ollama server
    \item \texttt{REMOTE}: Uses remote API
    \item \texttt{HUGGINGFACE}: Uses Hugging Face models
    \item \texttt{GITHUB}: Uses GitHub Copilot
\end{itemize}

\subsection{Dictionary Mode}

The package includes a dictionary mode for faster processing of common statements:

\begin{lstlisting}
# Enable dictionary mode
Config.CONFIG.dictionary_mode = :enabled

# Process common statements faster
oscar_code = process_statement("Find the roots of x^2 - 1")
\end{lstlisting}

\subsection{Debug Mode}

Debug mode provides detailed logging of the code generation process:

\begin{lstlisting}
# Enable debug mode
Config.CONFIG.debug = true

# See detailed logs
process_statement("Find the roots")
\end{lstlisting}

\section{Advanced Features}

\subsection{History Management}
\label{sec:history_management}
\indexmod{History}

The \modname{History} module provides functionalities to manage the log of interactions, including user inputs, generated Oscar code, and validation status. All history functions are accessible via the \modname{History} module (e.g., \code{History.add\_entry!}).

\subsubsection{The \code{HistoryEntry} Type}
\indexfunc{HistoryEntry}{type}
Most functions that retrieve history entries return or operate on objects of type \code{HistoryEntry}. This type has the following fields:
\begin{itemize}
    \item \code{timestamp::String}: The timestamp when the entry was created (e.g., \code{"2025-06-04\_18-10-11"}).
    \item \code{original\_statement::String}: The original natural language statement provided by the user.
    \item \code{generated\_code::String}: The Oscar code generated by the system.
    \item \code{is\_valid::Bool}: A boolean indicating whether the \code{generated\_code} was deemed valid by the validator.
    \item \code{validation\_error::Union{Nothing,String}}: If \code{is\_valid} is false, this field may contain a string describing the validation error. Otherwise, it is \code{nothing}.
\end{itemize}

\subsubsection{Core History Functions}

\paragraph{\func{History.add\_entry!}}
\indexfunc{History.add_entry!}
Adds a new entry to the history log.
\begin{lstlisting}[language=Julia, basicstyle=\footnotesize\ttfamily\color{black}]
function add_entry!(
    timestamp::String,
    original::String,
    generated::String,
    is_valid::Bool,
    validation_error::Union{Nothing,String} = nothing
)
\end{lstlisting}
\begin{itemize}
    \item \textbf{Arguments}: Details of the interaction to be logged. The \code{timestamp} can be generated using \code{History.current\_timestamp()}.
\end{itemize}

\paragraph{\func{History.get\_entries}}
\indexfunc{History.get\_entries}
Retrieves all history entries currently stored.
\begin{lstlisting}[language=Julia, basicstyle=\footnotesize\ttfamily\color{black}]
function get_entries()::Vector{HistoryEntry}
\end{lstlisting}
\begin{itemize}
    \item \textbf{Returns}: A vector of \code{HistoryEntry} objects.
\end{itemize}

\paragraph{\func{History.get\_entry}}
\indexfunc{History.get\_entry}
Retrieves a specific history entry by its 1-based index.
\begin{lstlisting}[language=Julia, basicstyle=\footnotesize\ttfamily\color{black}]
function get_entry(idx::Int)::HistoryEntry
\end{lstlisting}
\begin{itemize}
    \item \textbf{Arguments}: \code{idx} is the 1-based index of the entry to retrieve.
    \item \textbf{Returns}: The \code{HistoryEntry} object at the specified index.
    \item \textbf{Throws}: An error if \code{idx} is out of bounds.
\end{itemize}

\paragraph{\func{History.delete\_entry!}}
\indexfunc{History.delete_entry!}
Deletes a history entry by its 1-based index.
\begin{lstlisting}[language=Julia, basicstyle=\footnotesize\ttfamily\color{black}]
function delete_entry!(idx::Int)
\end{lstlisting}

\paragraph{\func{History.clear\_entries!}}
\indexfunc{History.clear_entries!}
Removes all entries from the history log.
\begin{lstlisting}[language=Julia, basicstyle=\footnotesize\ttfamily\color{black}]
function clear_entries!()
\end{lstlisting}

\paragraph{\func{History.edit\_entry!}}
\indexfunc{History.edit_entry!}
Modifies an existing history entry. The original timestamp is preserved, and the entry is marked as valid with no validation error.
\begin{lstlisting}[language=Julia, basicstyle=\footnotesize\ttfamily\color{black}]
function edit_entry!(idx::Int, original::String, generated::String)
\end{lstlisting}
\begin{itemize}
    \item \textbf{Arguments}: \code{idx} is the 1-based index; \code{original} and \code{generated} are the new statement and code respectively.
\end{itemize}

\subsubsection{Displaying History}

\paragraph{\func{History.display\_history}}
\indexfunc{History.display\_history}
Prints history entries to the console in a formatted way. This function has multiple methods:
\begin{itemize}
    \item \code{display\_history()}: Displays all entries in the current history log.
    \item \code{display\_history(idx::Int)}: Displays the single entry at the specified 1-based index.
    \item \code{display\_history(entries::Vector{HistoryEntry})}: Displays a provided vector of history entries.
\end{itemize}

\subsubsection{Saving and Loading History}

\paragraph{\func{History.save\_history}}
\indexfunc{History.save\_history}
Saves the current history log to a specified JSON file.
\begin{lstlisting}[language=Julia, basicstyle=\footnotesize\ttfamily\color{black}]
function save_history(filename::String)
\end{lstlisting}

\paragraph{\func{History.load\_history}}
\indexfunc{History.load\_history}
Loads history entries from a specified JSON file, replacing any entries currently in memory.
\begin{lstlisting}[language=Julia, basicstyle=\footnotesize\ttfamily\color{black}]
function load_history(filename::String)
\end{lstlisting}

\subsubsection{Utility Functions}

\paragraph{\func{History.current\_timestamp}}
\indexfunc{History.current\_timestamp}
Returns a string representation of the current date and time, formatted for use as a timestamp in history entries (e.g., \code{"yyyy-mm-dd\_HH-MM-SS"}).
\begin{lstlisting}[language=Julia, basicstyle=\footnotesize\ttfamily\color{black}]
function current_timestamp()::String
\end{lstlisting}

This set of functions allows for comprehensive management of the interaction history within \modname{OscarAICoder.jl}.

\subsection{Context Management}

Context management allows for better code generation by maintaining state:

\begin{lstlisting}
# Set context
Context.set("variable", value)

# Get context
value = Context.get("variable")
\end{lstlisting}

\section{Configuration}

The package can be configured through the \modname{Config} module:

\begin{lstlisting}
# Set default backend
Config.CONFIG.default_backend = REMOTE

# Set model
Config.CONFIG.model = "gpt-4"

# Enable training mode
Config.CONFIG.training_mode = true
\end{lstlisting}

\section{Validation}

The package includes validation to ensure generated code is correct:

\begin{lstlisting}
# Validate Oscar code
is_valid = validate_oscar_code(oscar_code)

# Get validation errors
errors = get_validation_errors(oscar_code)
\end{lstlisting}

\section{Support}
\label{sec:support}

\subsection{Documentation}
For detailed documentation, including:
\begin{itemize}
    \item Mathematical examples
    \item Advanced usage
    \item Configuration options
    \item Error handling
    \item Validation rules
    \item Backend-specific configurations
\end{itemize}

\subsection{Getting Help}
\begin{itemize}
    \item Check the documentation first
    \item Open an issue on the repository for bugs and feature requests
    \item For usage questions, check the examples section
    \item For technical support, contact the maintainers
\end{itemize}

\subsection{Common Issues}
\begin{itemize}
    \item Backend connection issues
    \item Invalid Oscar code generation
    \item Context management problems
    \item Configuration errors
\end{itemize}

\printindex

\section{Advanced Features}
\label{sec:advanced-features}

\subsection{Backend Configuration}
\label{subsec:backend_config}

\modname{OscarAICoder.jl} supports multiple backends for processing statements. Each backend has its own configuration options:

\subsubsection{Local Backend}
The local backend connects to a locally running Ollama server.

\begin{lstlisting}[language=Julia]
# Configure local backend
configure_local_backend(
    url="http://localhost:11434/api/generate",
    model="qwen2.5-coder",
    temperature=0.7,
    max_tokens=256
)
\end{lstlisting}

\subsubsection{Hugging Face Backend}
The Hugging Face backend uses the Hugging Face Inference API.

\begin{lstlisting}[language=Julia]
# Configure Hugging Face backend
configure_huggingface_backend(
    api_key="your_api_key",
    model="gpt2",
    endpoint="https://api-inference.huggingface.co/models"
)
\end{lstlisting}

\subsubsection{GitHub Backend}
The GitHub backend interacts with GitHub's API for code generation.

\begin{lstlisting}[language=Julia]
# Configure GitHub backend
configure_github_backend(
    repo="username/repo",
    token="your_github_token",
    branch="main",
    model="llama2"
)
\end{lstlisting}

\subsection{Dictionary Mode}
\label{subsec:dictionary_mode}

The package includes a dictionary-based approach for common mathematical operations. You can configure the dictionary mode using:

\begin{lstlisting}[language=Julia]
# Configure dictionary mode
configure_dictionary_mode(:priority)  # Options: :priority, :only, :disabled
\end{lstlisting}

\begin{description}
    \item[\code{:priority}] (default) Check dictionary first, then use LLM
    \item[\code{:only}] Only use dictionary, don't use LLM
    \item[\code{:disabled}] Always use LLM, ignore dictionary
\end{description}

\subsection{Debug and Training Modes}

\begin{lstlisting}[language=Julia]
# Enable debug mode for detailed logging
debug_mode!(true)

# Enable training mode for more detailed prompts
training_mode!(true)
\end{lstlisting}

\section{Examples}
\label{sec:examples}

Here are some examples of using \modname{OscarAICoder.jl} for various mathematical tasks:

\subsection{Basic Algebra}

\begin{lstlisting}[language=Julia]
# Solve a quadratic equation
code = process_statement("Find the roots of x^2 - 5x + 6")
println(code)

# Expected output:
# R, x = PolynomialRing(QQ, "x")
# f = x^2 - 5x + 6
# roots(f)
\end{lstlisting}

\subsection{Linear Algebra}

\begin{lstlisting}[language=Julia]
# Compute eigenvalues of a matrix
code = process_statement("Find the eigenvalues of matrix [1 2; 3 4]")
println(code)

# Expected output:
# M = matrix(QQ, [1 2; 3 4])
# eigvals(M)
\end{lstlisting}

\subsection{Number Theory}

\begin{lstlisting}[language=Julia]
# Check if a number is prime
code = process_statement("Check if 17 is a prime number")
println(code)

# Expected output:
# is_prime(17)  # Returns true
\end{lstlisting}

\section{API Reference}
\label{sec:api_reference}

This section provides a detailed reference for the core functions available in \modname{OscarAICoder.jl}.

\subsection{\func{process\_statement}}
\label{func:process_statement}

Processes a natural language mathematical statement and generates the corresponding Oscar code.

\subsubsection*{Signature}
\begin{lstlisting}[language=Julia]
function process_statement(statement::String; 
    backend::Config.BackendType=Config.CONFIG.default_backend,
    model::String="",
    api_key::Union{String,Nothing}=nothing,
    token::Union{String,Nothing}=nothing,
    use_history::Bool=true,
    clear_context::Bool=false,
    kwargs...
)
\end{lstlisting}

\subsubsection*{Parameters}
\begin{description}
    \item[\code{statement::String}] (Required) The natural language mathematical statement to process.
    \item[\code{backend::Config.BackendType}] (Optional) The backend to use for code generation. Defaults to \code{Config.CONFIG.default\_backend}. Possible values are \code{:LOCAL}, \code{:REMOTE}, \code{:HUGGINGFACE}, \code{:GITHUB}.
    \item[\code{model::String}] (Optional) The specific model to use for the selected backend. If empty, the default model for the backend is used.
    \item[\code{api\_key::Union\{String,Nothing\}}] (Optional) API key, required if using the \code{:HUGGINGFACE} backend with certain models. Defaults to \code{nothing}.
    \item[\code{token::Union\{String,Nothing\}}] (Optional) Authentication token, required if using the \code{:GITHUB} backend. Defaults to \code{nothing}.
    \item[\code{use\_history::Bool}] (Optional) Whether to use the existing conversation history/context for code generation. Defaults to \code{true}.
    \item[\code{clear\_context::Bool}] (Optional) Whether to clear the existing conversation history/context before processing the statement. Defaults to \code{false}.
    \item[\code{kwargs...}] (Optional) Additional backend-specific keyword arguments.
\end{description}

\subsubsection*{Returns}
\begin{itemize}
    \item Generated Oscar code as a \code{String}.
\end{itemize}

\subsubsection*{Throws}
\begin{itemize}
    \item \code{ErrorException} if the input statement is invalid or if code generation fails.
\end{itemize}


\subsection{\func{execute\_statement}}
\label{func:execute_statement}

Executes a string of Oscar code. It can optionally clear the current context before execution.

\subsubsection*{Signature}
\begin{lstlisting}[language=Julia]
function execute_statement(oscar_code::Union{String, SubString{String}}=""; clear_context=false)
\end{lstlisting}

\subsubsection*{Parameters}
\begin{description}
    \item[\code{oscar\_code::Union\{String, SubString\{String\}\}}] (Optional) The Oscar code string to execute. If empty, it might try to use code from history (behavior may vary). Defaults to an empty string.
    \item[\code{clear\_context::Bool}] (Optional) Whether to clear the current Oscar session's context before executing the code. Defaults to \code{false}.
\end{description}

\subsubsection*{Returns}
\begin{itemize}
    \item The result of the Oscar code execution. The type of the result depends on the executed code.
\end{itemize}

\subsubsection*{Throws}
\begin{itemize}
    \item \code{ErrorException} if the Oscar code is invalid or if execution fails.
\end{itemize}


\subsection{\func{execute\_statement\_with\_format}}
\label{func:execute_statement_with_format}

Executes a string of Oscar code and returns the result in a specified format.

\subsubsection*{Signature}
\begin{lstlisting}[language=Julia]
function execute_statement_with_format(oscar_code::String; output_format=:string)
\end{lstlisting}

\subsubsection*{Parameters}
\begin{description}
    \item[\code{oscar\_code::String}] (Required) The Oscar code string to execute.
    \item[\code{output\_format::Symbol}] (Optional) The desired format for the output. Currently supported: \code{:string} (default), which returns the output as a string. Other formats might be supported by the underlying Oscar system.
\end{description}

\subsubsection*{Returns}
\begin{itemize}
    \item The result of the Oscar code execution, formatted according to \code{output\_format}.
\end{itemize}

\subsubsection*{Throws}
\begin{itemize}
    \item \code{ErrorException} if the Oscar code is invalid, execution fails, or the requested output format is not supported.
\end{itemize}


\subsection{Configuration Functions}
\label{sec:config_functions}

These functions allow you to inspect and modify the global configuration of \modname{OscarAICoder.jl}.

\subsubsection{\func{set\_local\_model}}
\begin{lstlisting}[language=Julia]
function set_local_model(model_name::String)
\end{lstlisting}
Sets the model name for the local backend (e.g., Ollama). The specified model must be available on your local LLM server.

\subsubsection{\func{get\_local\_model}}
\begin{lstlisting}[language=Julia]
function get_local_model()
\end{lstlisting}
Returns the currently configured model name for the local backend as a \code{String}.

\subsubsection{\func{set\_training\_mode}}
\begin{lstlisting}[language=Julia]
function set_training_mode(mode::Bool)
\end{lstlisting}
Enables or disables training mode. When training mode is enabled, the system might behave differently, for example, by storing more detailed information about interactions. Pass \code{true} to enable, \code{false} to disable.

\subsubsection{\func{get\_training\_mode}}
\begin{lstlisting}[language=Julia]
function get_training_mode()
\end{lstlisting}
Returns \code{true} if training mode is enabled, \code{false} otherwise.

\subsubsection{\func{set\_dictionary\_mode}}
\begin{lstlisting}[language=Julia]
function set_dictionary_mode(mode::Symbol)
\end{lstlisting}
Sets the dictionary mode. The dictionary can be used to speed up responses for common queries. Possible values for \code{mode} are typically symbols like \code{:enabled}, \code{:disabled}, or \code{:partial}.

\subsubsection{\func{get\_dictionary\_mode}}
\begin{lstlisting}[language=Julia]
function get_dictionary_mode()
\end{lstlisting}
Returns the current dictionary mode as a \code{Symbol}.

\subsubsection{\func{set\_debug\_mode}}
\begin{lstlisting}[language=Julia]
function set_debug_mode(mode::Bool)
\end{lstlisting}
Enables or disables debug mode. When enabled, the system provides more verbose logging output, which can be helpful for troubleshooting. Pass \code{true} to enable, \code{false} to disable.

\subsubsection{\func{get\_debug\_mode}}
\begin{lstlisting}[language=Julia]
function get_debug_mode()
\end{lstlisting}
Returns \code{true} if debug mode is enabled, \code{false} otherwise.


\section{Troubleshooting}
\label{sec:troubleshooting}

\subsection{Common Issues}

\begin{description}
    \item[\important{LLM not responding}] Check if the LLM server is running and accessible. For local setups, ensure Ollama is properly installed and running.
    
    \item[\important{Context not maintained}] Ensure you're not clearing the context unintentionally and that the conversation history length is sufficient.
\end{description}

\subsection{Debugging}

You can enable debug mode to get more detailed information about the code generation process:

\begin{lstlisting}[language=Julia]
# Enable debug mode
debug_mode!(true)

# Process a statement to see debug output
process_statement("your statement")
\end{lstlisting}

% \section{API Reference}
% \label{sec:api_reference}

% \subsection{Main Functions}

% \begin{description}
%     \item[\func{process\_statement(statement::String; kwargs...)}] \hfill \\
%     Processes a natural language mathematical statement and returns the corresponding Oscar code.
%     \indexfunc{process\_statement}
    
%     \item[\func{configure\_default\_backend(backend::Symbol; kwargs...)}] \hfill \\
%     Configures the default LLM backend with the specified parameters.
%     \indexfunc{configure\_default\_backend}
    
%     \item[\func{configure\_dictionary\_mode(mode::Symbol)}] \hfill \\
%     Configures how the dictionary is used when processing statements.
%     \indexfunc{configure\_dictionary\_mode}
    
%     \item[\func{training\_mode!(enabled::Bool)}] \hfill \\
%     Enables or disables training mode, which provides more detailed prompts to the LLM.
%     \indexfunc{training\_mode!}
% \end{description}

% \subsection{Context Management}

% \begin{description}
%     \item[\func{get\_context()}] \hfill \\
%     Returns the current conversation context.
%     \indexfunc{get\_context}
    
%     \item[\func{clear\_context!()}] \hfill \\
%     Clears the current conversation context.
%     \indexfunc{clear\_context!}
    
%     \item[\func{save\_context(filename::String="")}] \hfill \\
%     Saves the current context to a file.
%     \indexfunc{save\_context}
    
%     \item[\func{load\_context(filename::String; clear\_current::Bool=true)}] \hfill \\
%     Loads a context from a file.
%     \indexfunc{load\_context}
    
%     \item[\func{set\_save\_dir(dir::String)}] \hfill \\
%     Sets the directory where session files will be saved.
%     \indexfunc{set\_save\_dir}
% \end{description}

% \subsection{Backend Functions}

% \begin{description}
%     \item[\func{configure\_local\_backend(;url, model, temperature, max\_tokens)}] \hfill \\
%     Configures the local Ollama backend.
%     \indexfunc{configure\_local\_backend}
    
%     \item[\func{configure\_huggingface\_backend(;api\_key, model, endpoint)}] \hfill \\
%     Configures the Hugging Face backend.
%     \indexfunc{configure\_huggingface\_backend}
    
%     \item[\func{configure\_github\_backend(;repo, token, branch, model)}] \hfill \\
%     Configures the GitHub backend.
%     \indexfunc{configure\_github\_backend}
% \end{description}

% \subsection{Debugging Functions}

% \begin{description}
%     \item[\func{debug\_mode!(state::Bool)}] \hfill \\
%     Enables or disables debug mode for detailed logging.
%     \indexfunc{debug\_mode!}
    
%     \item[\func{debug\_mode()}] \hfill \\
%     Returns the current debug mode state.
%     \indexfunc{debug\_mode}
    
%     \item[\func{debug\_print(msg::String; prefix="[DEBUG]" )}] \hfill \\
%     Prints a debug message if debug mode is enabled.
%     \indexfunc{debug\_print}
% \end{description}

% \subsection{Execution Functions}

% \begin{description}
%     \item[\func{execute\_statement(oscar\_code::String; clear\_context=false)}] \hfill \\
%     Executes Oscar code and returns the result.
%     \indexfunc{execute\_statement}
    
%     \item[\func{execute\_statement\_with\_format(oscar\_code::String; output\_format=:string)}] \hfill \\
%     Executes Oscar code and returns the result in the specified format.
%     \indexfunc{execute\_statement\_with\_format}
% \end{description}

\section{Best Practices}
\label{sec:best_practices}

\begin{itemize}
    \item Always start with dictionary mode enabled for common operations
    \item Use training mode to improve the dictionary with new mathematical concepts
    \item Keep debug mode enabled during development for better insights
    \item Regularly update the dictionary with new mathematical patterns
\end{itemize}

\bibliographystyle{plain}
\bibliography{references}

\section{Testing}
\label{sec:testing}

The package includes a comprehensive test suite organized by mathematical area:

\begin{itemize}
    \item \texttt{doc/tst/commutative\_algebra/}
    \item \texttt{doc/tst/algebraic\_geometry/}
    \item \texttt{doc/tst/calculus/}
    \item \texttt{doc/tst/linear\_algebra/}
\end{itemize}

To run the tests, use the \texttt{make test} command:

\begin{lstlisting}[language=bash]
make test
\end{lstlisting}

\section{Contributing}
\label{sec:contributing}
\index{Contributing}

Contributions to OscarAICoder.jl are welcome! Please see the \texttt{CONTRIBUTING.md} file for guidelines.

\section*{Acknowledgements}
\label{sec:acknowledgements}
\index{Acknowledgements}

We would like to express our gratitude to the following individuals and organizations for their contributions and support:

\begin{itemize}
    \item The OSCAR development team for creating and maintaining the OSCAR computer algebra system
    \item The Julia community for their support and contributions
    \item All contributors who have helped improve OscarAICoder.jl through bug reports, feature requests, and code contributions
    \item The developers of the various LLM backends that make this package possible
\end{itemize}

\section{License}
\label{sec:license}
\index{License}

OscarAICoder.jl is licensed under the MIT License.

\clearpage
%\section*{Index}\addcontentsline{toc}{section}{Index}
\printindex
\end{document}

