\documentclass[11pt,a4paper]{article}
\usepackage{amsmath}
\usepackage{amsfonts}
\usepackage{amssymb}
\usepackage{graphicx}
\usepackage{hyperref}
\usepackage{listings}
\usepackage{color}
\usepackage{url}
\usepackage{fancyhdr}
\usepackage{geometry}
\usepackage{makeidx}

\geometry{left=2cm,right=2cm,top=2cm,bottom=2cm}
\pagestyle{fancy}
\fancyhf{}
\fancyhead[L]{\leftmark}
\fancyhead[R]{\thepage}
\fancyfoot[C]{\thepage}
\renewcommand{\headrulewidth}{0.4pt}
\renewcommand{\footrulewidth}{0.4pt}

\definecolor{codegreen}{rgb}{0,0.6,0}
\definecolor{codegray}{rgb}{0.5,0.5,0.5}
\definecolor{codepurple}{rgb}{0.58,0,0.82}
\definecolor{backcolour}{rgb}{0.95,0.95,0.92}

\lstdefinestyle{mystyle}{
    backgroundcolor=\color{backcolour},   
    commentstyle=\color{codegreen},
    keywordstyle=\color{magenta},
    numberstyle=\tiny\color{codegray},
    stringstyle=\color{codepurple},
    basicstyle=\footnotesize,
    breakatwhitespace=false,         
    breaklines=true,                 
    captionpos=b,                    
    keepspaces=true,                 
    numbers=left,                    
    numbersep=5pt,                   
    showspaces=false,                
    showstringspaces=false,
    showtabs=false,                  
    tabsize=2
}

\lstset{style=mystyle}

\title{\textbf{OscarAICoder.jl}}
\author{Vinay Wagh}
\date{\today}

\begin{document}

\maketitle

\begin{abstract}
OscarAICoder.jl is a Julia package that translates English mathematical statements into Oscar code using various LLM backends. It supports multiple backend configurations including local, remote, GitHub-hosted, and HuggingFace models.
\end{abstract}

\tableofcontents

\section{Introduction}

OscarAICoder.jl is a Julia package designed to bridge the gap between natural language mathematical statements and executable Oscar code. It leverages modern LLM (Large Language Model) technology to understand and translate mathematical concepts expressed in English into equivalent Oscar code.

\section{Installation}

To install OscarAICoder.jl, use the Julia package manager:

\begin{lstlisting}[language=Julia]
using Pkg
Pkg.add("OscarAICoder")
\end{lstlisting}

\section{Basic Usage}

The primary function in OscarAICoder.jl is \texttt{process\_statement}, which takes an English mathematical statement and returns the corresponding Oscar code.

\subsection{Dictionary Mode}

The package includes a built-in dictionary of common mathematical statements. You can configure how it's used:

\begin{lstlisting}[language=Julia]
using OscarAICoder

# Try dictionary first, then LLM
configure_dictionary_mode(:priority)

# Use only the dictionary
configure_dictionary_mode(:only)

# Never use dictionary
configure_dictionary_mode(:disabled)
\end{lstlisting}

\section{Backend Configuration}

OscarAICoder supports multiple LLM backends. The default backend can be configured using \texttt{configure\_default\_backend}.

\subsection{Local Backend}

The local backend connects to a locally running LLM server (e.g., Ollama):

\begin{lstlisting}[language=Julia]
using OscarAICoder

# Configure local backend
configure_default_backend(:local)

# Process a statement
oscar_code = process_statement("Find the roots of x^2 - 5x + 6 = 0")
\end{lstlisting}

\subsection{Remote Backend}

The remote backend connects to a remote LLM server:

\begin{lstlisting}[language=Julia]
using OscarAICoder

# Configure remote backend
configure_default_backend(:remote, url="http://server01.mydomain.net:11434")

# Process a statement
oscar_code = process_statement("Find the roots of x^2 - 5x + 6 = 0")
\end{lstlisting}

\subsection{GitHub Backend}

The GitHub backend uses models hosted in GitHub repositories:

\begin{lstlisting}[language=Julia]
using OscarAICoder

# Configure GitHub backend
configure_github_backend(
    repo="username/repo",
    token="github_token",
    model="llama2"
)

# Process a statement
oscar_code = process_statement("Find the roots of x^2 - 5x + 6 = 0")
\end{lstlisting}

\section{Examples}

\subsection{Basic Algebra}

\subsubsection{Polynomial Factorization}

\begin{lstlisting}[language=Julia]
using OscarAICoder

# Configure dictionary mode
configure_dictionary_mode(:priority)

# Process statement
oscar_code = process_statement("Factor the polynomial x^2 - 5x + 6 over the integers")
\end{lstlisting}

\subsection{Commutative Algebra}

\subsubsection{Ideal Operations}

\begin{lstlisting}[language=Julia]
using OscarAICoder

# Configure GitHub backend
configure_github_backend(
    repo="username/repo",
    token="github_token",
    model="llama2"
)

# Process statement
oscar_code = process_statement("Compute the intersection of ideals (x,y) and (y,z) in Q[x,y,z]")
\end{lstlisting}

\subsection{Algebraic Geometry}

\subsubsection{Variety Computation}

\begin{lstlisting}[language=Julia]
using OscarAICoder

# Configure dictionary mode
configure_dictionary_mode(:priority)

# Process statement
oscar_code = process_statement("Compute the variety defined by x^2 + y^2 - 1 in Q[x,y]")
\end{lstlisting}

\subsection{Calculus}

\subsubsection{Integration}

\begin{lstlisting}[language=Julia]
using OscarAICoder

# Configure local backend
configure_default_backend(:local)

# Process statement
oscar_code = process_statement("Integrate x^2 from 0 to 1")
\end{lstlisting}

\subsection{Linear Algebra}

\subsubsection{Matrix Operations}

\begin{lstlisting}[language=Julia]
using OscarAICoder

# Configure dictionary mode
configure_dictionary_mode(:priority)

# Process statement
oscar_code = process_statement("Compute the determinant of matrix [[1,2],[3,4]]")
\end{lstlisting}

\section{Advanced Features}

\subsection{Offline Mode}

You can enable offline mode to use only the local dictionary:

\begin{lstlisting}[language=Julia]
using OscarAICoder

# Enable offline mode
configure_offline_mode(true)

# Only dictionary entries will be used
oscar_code = process_statement("Compute the determinant of a 2x2 matrix")
\end{lstlisting}

\subsection{Backend Configuration}

\subsubsection{Local Backend}

The local backend connects to a locally running LLM server (e.g., Ollama):

\begin{lstlisting}[language=Julia]
using OscarAICoder

# Configure local backend
configure_default_backend(:local)

# Process a statement
oscar_code = process_statement("Find the roots of x^2 - 5x + 6 = 0")
\end{lstlisting}

\subsubsection{Remote Backend}

The remote backend connects to a remote LLM server:

\begin{lstlisting}[language=Julia]
using OscarAICoder

# Configure remote backend
configure_default_backend(:remote, url="http://server01.mydomain.net:11434")

# Process a statement
oscar_code = process_statement("Find the roots of x^2 - 5x + 6 = 0")
\end{lstlisting}

\subsubsection{GitHub Backend}

The GitHub backend uses models hosted in GitHub repositories:

\begin{lstlisting}[language=Julia]
using OscarAICoder

# Configure GitHub backend
configure_github_backend(
    repo="username/repo",
    token="github_token",
    model="llama2"
)

# Process a statement
oscar_code = process_statement("Find the roots of x^2 - 5x + 6 = 0")
\end{lstlisting}

\section{Testing}

The package includes a comprehensive test suite organized by mathematical area:

\begin{itemize}
    \item \texttt{doc/tst/commutative\_algebra/}
    \item \texttt{doc/tst/algebraic\_geometry/}
    \item \texttt{doc/tst/calculus/}
    \item \texttt{doc/tst/linear\_algebra/}
\end{itemize}

To run the tests, use the \texttt{make test} command:

\begin{lstlisting}[language=bash]
make test
\end{lstlisting}

\section{Contributing}

Contributions to OscarAICoder.jl are welcome! Please see the \texttt{CONTRIBUTING.md} file for guidelines.

\section{License}

OscarAICoder.jl is licensed under the MIT License.

\end{document}
